\documentclass[12pt,a4paper]{article}
\usepackage{amsmath, amssymb, amsthm}
\usepackage{qdichokey} 
% This sty. is download from https://github.com/Mikumikunisiteageru/qdichokey
\setlength{\parindent}{2.05em} %中文兩字大小空格;總編輯時可能會再做修改
\usepackage[shortlabels]{enumitem}
\usepackage{verbatim} %for comment
\setlist[enumerate]{resume}
\usepackage{graphics}
\usepackage[margin=3cm]{geometry}
\usepackage{fontspec}
\usepackage{xeCJK}
\setCJKmainfont[AutoFakeBold=3]{[TWKai.ttf]} %自訂標楷體
\XeTeXlinebreaklocale "zh"             
\XeTeXlinebreakskip = 0pt plus 1pt
\usepackage{comment}

% Some tips here.<-
% Codes above are the environment setting , omit it.
% In qdichokey.sty, we use Key* instead of Key. 
% \section*{} creates a section without numbering.
% \begin{Key*}{}  and \end{Key*} to create a new column.
% Behind \alter, type the descriptions.
% Behind \name, type the name of the species.
% \testit{} tilt the text.


\begin{document}

\large 本地樹木樟科自製檢索表(作者:黃敬麟、陳睿騏、殷楷智)\\
\vspace{2ex}
\normalsize


\subsection*{屬檢索表}
\begin{Key*}{}
\indent\alter 三出脈
    \alter 叢生枝頂或近輪生狀
    \cname{9.新木薑子屬}{Neolitsea}
    \alter 互生或近對生
        \alter 葉先端長尾狀
        \cname{6.釣樟屬}{Lindera}
        \alter 葉先端不是長尾狀
            \alter 明顯二到三對二級脈且葉脈腋處有蟲室,或只有一對明顯二級脈
            \cname{2.樟屬}{Cinnamomum}
            \alter 明顯二到三對二級脈且葉脈腋處無蟲室
                \alter 離基三出脈前有一對細脈自中肋伸出
                \cname{2.樟屬}{Cinnamomum}
                \alter 離基三出脈前無一對細脈自中肋伸出
                \cname{3.厚殼桂屬}{Cryptocarya}
            
\alter 羽狀脈
    \alter 叢生枝頂或近輪生狀
        \alter 葉柄有毛
            \alter 葉無毛
            \cname{9.新木薑子屬}{Neolitsea}
            \alter 葉有毛
                \alter 靠葉基部的二級脈與中肋垂直(木薑子脈)
                \cname{7.木薑子屬}{Litsea}
                \alter 靠葉基部的二級脈沒有與中肋垂直
                    \alter 三級脈明顯凸起
                    \cname{10.雅楠屬}{Phoebe}
                    \alter 三級脈無明顯凸起
                    \cname{6.釣樟屬}{Lindera}
        \alter 葉柄無毛
            \alter 葉柄長於3公分
            \cname{11.檫樹屬}{Sassafras}
            \alter 葉柄小於3公分
                \alter 葉柄小於0.8公分
                \cname{8.楨楠屬}{Machilus}
                \alter 葉柄大於0.8公分
                    \alter 葉背有毛或枝條皮孔明顯
                    \cname{6.釣樟屬}{Lindera}
                    \alter 葉無毛且枝條無皮孔
                    \cname{4.腰果楠屬}{Dehaasia}
    \alter 互生或近對生
        \alter 葉脈腋有蟲室
        \cname{3.樟屬}{Cinnamomum}
        \alter 葉脈腋無蟲室
            \alter 靠葉基部的二級脈與中勒垂直(木薑子脈)
                \cname{7.木薑子屬}{Litsea}
            \alter 靠葉基部的二級脈沒有與中勒垂直
                \alter 葉子正反兩面顏色相同
                \cname{1.瓊楠屬}{Beilschmiedia}
                \alter 葉子正反兩面顏色不相同
                    \alter 葉正面中肋凹
                        \alter 成熟葉葉背有毛
                            \alter 葉先端鈍、圓或短凸,或葉卵形至倒卵形
                            \cname{7.木薑子屬}{Litsea}
                            \alter 葉先端銳尖至尾狀(但非短凸),且葉為倒披針至披針形
                            \cname{8.楨楠屬}{Machilus}
                        \alter 成熟葉無毛
                            \alter 側脈7對以上
                            \cname{8.楨楠屬}{Machilus}
                            \alter 側脈7對以下
                            \cname{3.厚殼桂屬}{Cryptocarya}
                    \alter 葉正面中肋凸
                        \alter 葉背無毛
                            \alter 葉背淺綠色呈白粉狀
                            \cname{8.楨楠屬}{Machilus}
                            \alter 葉背亮綠色有光澤
                            \cname{5.三蕊楠屬}{Endiandra}
                        \alter 葉背有毛
                            \alter 葉正面有網狀脈
                            \cname{1.瓊楠屬}{Beilschmiedia}
                            \alter 葉正面無網狀脈
                                \alter 全株有馬告味或葉柄大於1公分
                                \cname{7.木薑子屬}{Litsea}
                                \alter 全株無馬告味且葉柄小於1公分
                                \cname{6.釣樟屬}{Lindera}
\end{Key*}
%\vspace{5ex}
\newpage
\subsection*{種檢索表}
\section{瓊楠屬\textit{Beilschmiedia}}
\begin{Key*}{B.~}
\indent\alter 葉背深綠或葉正反兩面顏色相同
\cname{瓊楠}{erythrophloia}
\alter 葉背淺綠或葉正反兩面顏色不同
\cname{華河瓊楠 (廣東瓊楠,毛瓊楠)}{tsangii}
\end{Key*}

\section{樟屬\textit{Cinnamomum}}

\begin{Key*}{C.~}  
\indent\alter 羽狀脈 
    \alter 二級脈分岔少或枝條有圓形皮孔
    \cname{冇樟}{micranthum}
    \alter 二級脈有明顯分岔,枝條無圓形皮孔
    \cname{牛樟}{kanehirae}
\alter 三出脈
    \alter 三出脈直達葉先端
        \alter 葉背灰白且有毛
        \cname{牡丹葉桂皮}{austrosinense}
        \alter 葉背淡綠且無毛
            \alter 葉卵形
            \cname{小葉樟}{brevipedunculatum}
            \alter 葉長橢圓至披針形形
             \cname{香桂}{subavenium}
    \alter 三出脈無直達葉先端
        \alter 先端圓或鈍
        \cname{土樟}{reticulatum}
	    \alter 先端尾狀、短凸、漸尖或銳尖
		    \alter 葉背綠或淺綠
		        \alter 葉卵、披針形且厚革質
		            \alter 基部三出脈,且葉無味
		            \cname{蘭嶼肉桂}{kotoense}
		            \alter 離基三出脈,且葉有泡泡糖味
		            \cname{錫蘭肉桂}{verum}
		        \alter 葉長橢圓、橢圓、披針、卵或闊卵形且革質
		            \alter 葉脈腋無蟲室,且葉有痱子粉味
		            \cname{陰香}{burmannii}
		            \alter 葉脈腋有蟲室
		                \alter 二級脈分岔少或枝條有圓形皮孔
                        \cname{冇樟}{micranthum}
                        \alter 二級脈有明顯分岔,枝條無圓形皮孔
                        \cname{牛樟}{kanehirae}
		    \alter 葉背白
		        \alter 葉脈腋有蟲室,且葉有樟腦味
		            \cname{樟樹}{camphora}
		        \alter 葉脈腋無蟲室
		            \alter 葉背有絨毛
		            \cname{天竺桂}{tenuifolium \f nervosum}
		            \alter 葉背無絨毛
		                \alter 裸芽,無鱗芽痕
		                \cname{土肉桂}{osmophloeum}
		                \alter 鱗芽,無鱗芽痕
		                    \alter 葉背除三出脈外的二級脈只出現在葉先端,且葉有檸檬味
		                    \cname{胡氏肉桂}{macrostemon}
		                    \alter 葉背除三出脈外的二級脈在中間的位置開始出現,且葉有蘋果味
		                    \cname{台灣肉桂}{insulari-montanum }
\end{Key*}
\newpage

\section{厚殼桂屬 \textit{Cryptocarya}}

\begin{Key*}{C.~} 
\indent\alter 葉背蒼白色
\cname{厚殼桂}{chinensis}
\alter 葉背淡綠色
    \alter 葉基銳尖
        \cname{土楠(海南厚殼桂)}{concinna}
    \alter 葉基鈍
        \cname{菲律賓厚殼桂}{elliptifolia}
\end{Key*}



\section{腰果楠屬\textit{Dehaasia}}
\begin{Key*}{D.~}
\indent\alter
\cname{腰果楠}{incrassata }
\end{Key*}

\section{三蕊楠屬\textit{Endiandra}}
\begin{Key*}{E.~}
\indent\alter
\cname{三蕊楠}{coriacea }
\end{Key*}

\section{釣樟屬 \textit{Lindera}}
\begin{Key*}{L.~}
\indent\alter 基出三出脈
\cname{天台烏藥}{ aggregata}
\alter 羽狀脈
    \alter 葉柄有毛
        \alter 葉短於4公分
        \cname{內苳子}{akoensis}  
        \alter 葉長於4公分
            \alter 葉背具橘黃色柔毛,葉橢圓至倒披針
            \cname{香葉樹}{communis}
            \alter 葉背脈被軟毛,葉橢圓至長橢圓
            \cname{白葉釣樟}{glauca}
    \alter 葉柄無毛 
        \alter 葉基漸狹,略延伸至葉柄兩側%或是用葉背有毛、葉大小來分
        \cname{鐵釘樹}{erythrocarpa}
        \alter 葉基楔形或不明顯漸狹且葉大
        \cname{大葉釣樟(大香葉樹)}{megaphylla }
\end{Key*}

\section{木薑子屬 \textit{Listea}}
\begin{Key*}{L.~}
\indent\alter 木薑子脈型
    \alter 葉肉質且大
    \cname{蘭嶼木薑子}{garciae}
    \alter 葉非肉質
        \alter 密佈灰褐色長柔毛
        \cname{霧社木薑子}{elongata \var mushaensis}
        \alter 葉與枝條的毛略不發達
            \alter 葉革質 
                \alter 有長葉尾漸尖
                \cname{能漢木薑子}{lii \var nunkao-tahangensis}
                \alter 葉短尾銳尖
                \cname{玉山木薑子}{morrisonensis }
            \alter 葉紙質至薄革質
                \alter 葉具明顯波狀緣
                    \alter 葉柄長於2公分
                    \cname{長葉木薑子}{acuminata}
                    \alter 葉柄1公分以下
                    \cname{銳脈木薑子}{acutivena}
                \alter 葉緣略呈波浪狀
                    \alter 遠軸面二級脈明顯凸起且黃
                    \cname{小梗木薑子}{hypophaea}
                    \alter 遠軸面二級脈無明顯凸起
                    \cname{李氏木薑子}{lii}
\alter 羽狀脈
    \alter 有馬告味
    \cname{山胡椒}{cubeba}
    \alter 沒有馬告味
        \alter 葉倒卵橢圓形至長橢圓形
            \alter 葉先漸尖
                \alter 遠軸面二級脈明顯凸起且黃
                \cname{佐佐木氏木薑子}{akoensisa \var sasakii}
                \alter 遠軸面二級脈無明顯凸起
                \cname{鹿皮斑木薑子}{coreana}
            \alter 葉先非漸尖
                \alter 葉先端圓鈍
                \cname{潺槁樹}{glutinosa}
                \alter 葉先端非圓形
                    \alter 葉背灰白
                    \cname{橢圓葉木薑子}{rotundifolia \var oblongifolia}
                    \alter 葉背灰褐
                    \cname{屏東木薑子}{akoensisa \var akoensisa}
        \alter 葉闊卵形
            \alter 無葉尾
            \cname{竹頭角木薑子}{akoensisa \var chitouchiaoensis}
            \alter 有葉尾
            \cname{佩羅特木薑子 (白達木. 菲律賓木薑子)}{perrottetii}
                
            
\end{Key*}{}

        
\section{楨楠屬 \textit{Machilus}}
\begin{Key*}{M.~}
\indent\alter 側脈8對以內
    \alter 葉叢生或輪生,且厚革質
        \alter 葉先端圓至鈍
        \cname{恆春楨楠}{obovatifolia \var obovatifolia}
        \alter 葉先端銳尖\\
        \cname{大武楨楠}{obovatifolia \var taiwuensis}
    \alter 葉互生,且革質或薄革質
        \alter 葉背中肋及側脈明顯被毛,葉偶反捲\\
        \cname{小西氏楠}{konishii}
        \alter 葉光滑或不明顯被毛、有波浪緣或三級脈亂長成格子
        \cname{菲律賓楠}{philippinensis}
\alter 側脈8對或以上
    \alter 葉背淡綠
        \alter 葉表面無光澤
            \cname{香楠}{zuihoensis \var zuihoensis}
        \alter 葉表面有光澤
            \alter 枝條白,皮孔少且光滑,葉柄短於1.5公分
            \cname{大葉楠}{japonica \var kusanoi}
            \alter 枝條有明顯皮孔,且不光滑,葉柄長於1.5公分
            \cname{霧社楨楠}{zuihoensis \var mushaensis}
    \alter 葉背灰白、蒼白或粉綠
        \alter 葉先端短凸或短漸尖,或三級脈互相平行排列整齊
            \cname{紅楠}{thunbergii}
        \alter 葉先端銳尖、鈍、短尾狀或尾狀[5/7]
             \alter 無白色木質化枝條%枝條皮孔明顯,綠色
             \cname{霧社楨楠}{zuihoensis \var mushaensis}
             \alter 有白色木質化枝條,枯葉帶有紫暈%枝條皮孔不明顯,白色,且光滑[皮孔目前覺得很明顯]
             \cname{假長葉楠}{japonica \var japonica}
    
\end{Key*}

\section{新木薑子屬 \textit{Neolitsea}}
\begin{Key*}{N.~}
\indent\alter 極少三出脈,多為羽狀脈
\cname{高山新木薑子}{acuminatissima}
\alter 離基三出脈
    \alter 近輪生於枝頂一點
        \alter 葉尖先端漸尖或突尖
        \cname{五掌楠}{konishii}
        \alter 葉尖先端圓鈍
            \alter 小枝被褐色伏毛
            \cname{武威新木薑子}{buisanensis}
            \alter 小枝綠色無毛
            \cname{南仁新木薑子}{hiiranensis}
    \alter 近輪生或叢生於小枝頂端前1/2或1/4以上
        \alter 葉背無毛
            \alter 互生於枝條前半段,有龍眼味
            \cname{小芽新木薑子}{parvigemma}
            \alter 叢生於枝條前1/4以上,無龍眼味
            \cname{大武新木薑子}{daibuensis}
        \alter 葉背有毛
            \alter 葉背宿存褐色或金色毛
                \alter 葉背金色
                \cname{金新木薑子}{sericea \var aurata}
                \alter 葉背白色
                    \alter 葉背白色輕搓易落,葉芽長橢圓形
                    \cname{銳葉新木薑子}{aciculata}
                    \alter 葉背白色輕搓不易落,葉芽闊橢圓形
                    \cname{白新木薑子}{sericea \var sericea}
            \alter 葉背白色或褐色毛易落
                \alter 小枝密被褐色長毛
                \cname{蘭嶼新木薑子}{villosa}
                \alter 小枝幾無毛或易落短毛
                \cname{變葉新木薑子}{aciculata \var viabilima}

\end{Key*}


\section{雅楠屬\textit{Phoebe}}
\begin{Key*}{P.~}
\indent\alter
\cname{台灣雅楠}{formosana }
\end{Key*}


\section{檫樹屬\textit{Sassafras}}
\begin{Key*}{S.~}
\indent\alter
\cname{台灣檫樹}{randaiense }
\end{Key*}

%Test for \cname and \var and \f
\begin{comment}
\begin{Key*}{A.~}
\indent\alter
\cname{前面}{\var B}
\alter
\cname{front}{\f B}
\end{Key*}
\end{comment}



\end{document}
